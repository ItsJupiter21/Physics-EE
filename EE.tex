\documentclass[a4paper,12pt]{article}
\usepackage[utf8]{inputenc}
\usepackage[style=authoryear,backend=biber]{biblatex}

\renewcommand{\baselinestretch}{2}
\renewcommand*\contentsname{Table Of Contents}
\addbibresource{bibliography.bib}
\begin{document}

\begin{titlepage}
    \begin{center}
        \vspace*{1cm}

        \textbf{Investigating Factors Affecting the Resonant Frequency of Cantilever Beams}\\

        \vspace{.5cm}
        \textbf{Research Question:}
        To what extent does changing the properties of a circular cantilever beam affect its resonant frequency, and how well can the theoretical model predict this relationship?

        \vspace{0.5cm}
        International Baccalaureate Physics Extended Essay

        \vfill

        \vspace{0.8cm}

        Word Count:210


    \end{center}
\end{titlepage}

\pagenumbering{Roman}

% Table of Contents
\tableofcontents
\addcontentsline{toc}{section}{Table Of Contents}
\pagebreak

\begin{abstract}
\addcontentsline{toc}{section}{Abstract}

    Write something here

\end{abstract}
\pagebreak

\pagenumbering{arabic}

\section{Introduction}%500

    Cantilever beams, seemingly simple elements used in construction where one end of a beam is solidly attached and the other loose, has played a major role in engineering, historically they were used for purely mechanical structures such as buildings, cranes and balconies.
    \autocite{BuildingConstructionBook}\\
    While the use of such simple elements is still common in archaic construction, their utility expands far beyond macroscopic architecture, Nowadays cutting edge technology such as MEMS(micro-electromechanical) systems which bridge electrical and mechanical systems on the semiconductor level, use such structures as well, microscopic strain gauges being used in gyroscopes to sense tiny acceleration forces.
    \autocite{MemsBook}%Analysis and Design Principles of MEMS Devices


    Personally, I was impressed that such simple structures was in the core of such advanced technology we take for granted today, gyroscopes and accelerometers are the core of car safety systems and mobile device tracking, and used by billions everyday.

        The research question is `` To what extent does changing the properties of a circular cantilever beam affect its resonant frequency, and how well can the theoretical model predict this relationship? '' and I aim to

    I aim to investigate how the length of the beam and the initial energy loaded into the beam will affect the frequency that the beam naturally vibrates at, or the resonant frequency, according to the theory,  the length of the beam should affect the frequency but the initial energy transferred should not affect it.
    I specifically picked this topic to write about to because I was drawn to a simple topic having such complex calculations and wide use cases.

\section{Background Information}%600
    \subsection{Fourier Transform}

\section{Methodology}%700
    \subsection{Theory}

    \subsection{Materials}


\section{Results And Analysis}%500
    \subsection{Uncertainty Analysis}

\section{Discussion}%500

\section{Conclusion}%500

\printbibliography
\end{document}
